% Code examples section
\section{Code Examples}
\label{sec:code-examples}

This section demonstrates syntax highlighting for different programming languages.

\subsection{C++ Code Example}
\label{sec:cpp-example}

Here's an example of a C++ class with various language features:

\begin{lstlisting}[style=cpp, caption={C++ Example: Simple Vector Class}]
#include <iostream>
#include <vector>
#include <memory>

// Template class for a simple 2D vector
template<typename T>
class Vector2D {
private:
    T x, y;
    
public:
    // Constructor
    Vector2D(T x_val, T y_val) : x(x_val), y(y_val) {}
    
    // Copy constructor
    Vector2D(const Vector2D& other) = default;
    
    // Calculate magnitude
    double magnitude() const {
        return std::sqrt(x * x + y * y);
    }
    
    // Overload addition operator
    Vector2D operator+(const Vector2D& other) const {
        return Vector2D(x + other.x, y + other.y);
    }
    
    // Print vector
    void print() const {
        std::cout << "Vector(" << x << ", " << y << ")" << std::endl;
    }
};

int main() {
    // Create vectors using smart pointers
    auto v1 = std::make_shared<Vector2D<double>>(3.0, 4.0);
    auto v2 = std::make_shared<Vector2D<double>>(1.0, 2.0);
    
    // Perform operations
    auto v3 = *v1 + *v2;
    v3.print();  // Output: Vector(4, 6)
    
    std::cout << "Magnitude: " << v3.magnitude() << std::endl;
    
    return 0;
}
\end{lstlisting}

\subsection{Java Code Example}
\label{sec:java-example}

Here's an example of a Java class demonstrating object-oriented programming:

\begin{lstlisting}[style=java, caption={Java Example: Bank Account System}]
import java.util.*;

// Abstract class for bank accounts
public abstract class BankAccount {
    private String accountNumber;
    private String owner;
    protected double balance;
    
    // Constructor
    public BankAccount(String accountNumber, String owner, double initialBalance) {
        this.accountNumber = accountNumber;
        this.owner = owner;
        this.balance = initialBalance;
    }
    
    // Abstract method for interest calculation
    public abstract double calculateInterest();
    
    // Deposit money
    public void deposit(double amount) {
        if (amount > 0) {
            balance += amount;
            System.out.println("Deposited: $" + amount);
        }
    }
    
    // Withdraw money
    public boolean withdraw(double amount) {
        if (amount > 0 && amount <= balance) {
            balance -= amount;
            System.out.println("Withdrawn: $" + amount);
            return true;
        }
        return false;
    }
    
    // Get balance
    public double getBalance() {
        return balance;
    }
}

// Savings account implementation
class SavingsAccount extends BankAccount {
    private static final double INTEREST_RATE = 0.03;
    
    public SavingsAccount(String accountNumber, String owner, double initialBalance) {
        super(accountNumber, owner, initialBalance);
    }
    
    @Override
    public double calculateInterest() {
        return balance * INTEREST_RATE;
    }
}

// Main class
public class Main {
    public static void main(String[] args) {
        // Create a savings account
        SavingsAccount account = new SavingsAccount("SA001", "John Doe", 1000.0);
        
        // Perform operations
        account.deposit(500.0);
        account.withdraw(200.0);
        
        // Calculate and display interest
        double interest = account.calculateInterest();
        System.out.println("Interest: $" + interest);
        System.out.println("Balance: $" + account.getBalance());
    }
}
\end{lstlisting}

\subsection{Inline Code}
\label{sec:inline-code}

You can also use inline code snippets like \lstinline[style=cpp]{std::vector<int> nums;} for C++ or \lstinline[style=java]{ArrayList<String> list = new ArrayList<>();} for Java within your text.

